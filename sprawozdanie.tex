\documentclass{article}
\usepackage{graphicx}
\usepackage{amsmath}
\usepackage{polski}
\usepackage[utf8]{inputenc}

\title{Eksperymenty z całkowaniem Monte Carlo}
\author{Jakub Kogut}
\date{}

\begin{document}

\maketitle

\section{Wstęp}
Sprawozdanie do zadania domowego 1.

\section{Opis Zadania}
Dane są następujące całki, które należy przybliżyć:
\begin{itemize}
    \item \( \int_{0}^{8} 3 \sqrt{x} \, dx \)
    \item \( \int_{0}^{\pi} \sin(x) \, dx \)
    \item \( \int_{0}^{1} 4x(1 - x)^3 \, dx \)
\end{itemize}

Dla każdej z tych całek przeprowadzono eksperymenty, wykonując algorytm Monte Carlo dla \( n = 50, 100, \ldots, 5000 \) z \( k = 5 \) oraz \( k = 50 \) niezależnymi powtórzeniami. Eksperymenty obejmowały również aproksymację liczby \(\pi\) przy użyciu tej samej metody Monte Carlo.

\section{Metodologia}
Metoda całkowania Monte Carlo została przeprowadzona w następujący sposób:
\begin{enumerate}
    \item Generowano losowe punkty w obrębie prostokąta obejmującego wykres funkcji całkowanej.
    \item Zliczano punkty, które znalazły się pod wykresem funkcji.
    \item Stosunek punktów pod wykresem do całkowitej liczby punktów użyto do oszacowania wartości całki.
\end{enumerate}

\section{Wyniki}
Wyniki eksperymentów przedstawiono na poniższych wykresach, gdzie:
\begin{itemize}
    \item Niebieskie punkty reprezentują wyniki poszczególnych powtórzeń.
    \item Czerwone kropki przedstawiają średnią wartość dla każdego \( n \).
    \item Zielona linia oznacza dokładną wartość całki lub \(\pi\).
\end{itemize}

\begin{figure}[h]
    \centering
    \includegraphics[width=\textwidth]{plot_f1_5.png}
    \caption{Całka \( \int_{0}^{8} 3 \sqrt{x} \, dx \) dla \(k = 5\) powtórzeń}
\end{figure}

\begin{figure}[h]
    \centering
    \includegraphics[width=\textwidth]{plot_f1_50.png}
    \caption{Całka \( \int_{0}^{8} 3 \sqrt{x} \, dx \) dla \(k = 50\) powtórzeń}
\end{figure}

\begin{figure}[h]
    \centering
    \includegraphics[width=\textwidth]{plot_f2_5.png}
    \caption{Całka \( \int_{0}^{\pi} \sin(x) \, dx \) dla \(k = 5\) powtórzeń}
\end{figure}

\begin{figure}[h]
    \centering
    \includegraphics[width=\textwidth]{plot_f2_50.png}
    \caption{Całka \( \int_{0}^{\pi} \sin(x) \, dx \) dla \(k = 50\) powtórzeń}
\end{figure}

\begin{figure}[h]
    \centering
    \includegraphics[width=\textwidth]{plot_f3_5.png}
    \caption{Całka \( \int_{0}^{1} 4x(1 - x)^3 \, dx \) dla \(k = 5\) powtórzeń}
\end{figure}

\begin{figure}[h]
    \centering
    \includegraphics[width=\textwidth]{plot_f3_50.png}
    \caption{Całka \( \int_{0}^{1} 4x(1 - x)^3 \, dx \) dla \(k = 50\) powtórzeń}
\end{figure}

\begin{figure}[h]
    \centering
    \includegraphics[width=\textwidth]{plot_pi_50.png}
    \caption{Aproksymacja liczby \(\pi\) dla \(k = 50\) powtórzeń}
\end{figure}

\begin{figure}[h]
    \centering
    \includegraphics[width=\textwidth]{plot_pi_5.png}
    \caption{Aproksymacja liczby \(\pi\) dla \(k = 5\) powtórzeń}
\end{figure}


\end{document}
